\documentclass[11pt,aspectratio=169]{beamer}
 
\usetheme[sectionpage=none, subsectionpage=none, progressbar=none]{metropolis}           % Use metropolis theme
 
  \usepackage{outlines}
  \usepackage{caption}
  \usepackage{appendixnumberbeamer}
  \usepackage{booktabs}
  \usepackage{tcolorbox}
  \usepackage{tabularx}
  \usepackage[export]{adjustbox}[2011/08/13]
  \usepackage{bm}
  \usepackage{array}
  \usepackage{scalerel}
  \usepackage{xcolor}
  
  \def\msquare{\mathord{\scalerel*{\Box}{gX}}}

  
  \def\shrug{\texttt{\raisebox{0.75em}{\char`\_}\char`\\\char`\_\kern-0.5ex(\kern-0.25ex\raisebox{0.25ex}{\rotatebox{45}{\raisebox{-.75ex}"\kern-1.5ex\rotatebox{-90})}}\kern-0.5ex)\kern-0.5ex\char`\_/\raisebox{0.75em}{\char`\_}}}


\newcolumntype{R}{>{\raggedleft\arraybackslash}X} 

\newtheorem*{defn}{Def'n}

\DeclareUnicodeCharacter{2212}{-}

  
 \setbeamertemplate{footline}[frame number]{}

\setbeamertemplate{footline}{% 
  \hfill% 
  \usebeamercolor[fg]{page number in head/foot}% 
  \usebeamerfont{page number in head/foot}% 
  \insertframenumber%
  %\,/\,\inserttotalframenumber
  \kern1em\vskip2pt% 
}

\makeatletter
\setbeamertemplate{headline}{%
  \begin{beamercolorbox}[colsep=1.5pt]{upper separation line head}
  \end{beamercolorbox}
  \begin{beamercolorbox}{section in head/foot}
    \vskip2pt\insertnavigation{\paperwidth}\vskip2pt
  \end{beamercolorbox}%
  \begin{beamercolorbox}[colsep=1.5pt]{lower separation line head}
  \end{beamercolorbox}
}
\let\@@magyar@captionfix\relax % IMPORTANT: This is a workaround to fix a random eror with the 2018 installation
\makeatother

\usepackage{xcolor} 
\listfiles

\setbeamercolor{section in head/foot}{fg=normal text.bg, bg=structure.fg}

    \usepackage{smartdiagram}
    \usepackage{tikz}
\usetikzlibrary{shapes.geometric, arrows}
\tikzstyle{startstop} = [rectangle, rounded corners, minimum width=3cm, minimum height=1cm,text centered, draw=black, fill=red!30]
\tikzstyle{io} = [trapezium, trapezium left angle=70, trapezium right angle=110, minimum width=3cm, minimum height=1cm, text centered, draw=black, fill=blue!30]
\tikzstyle{process} = [rectangle, minimum width=3cm, minimum height=1cm, text centered, draw=black, fill=orange!30]
\tikzstyle{decision} = [diamond, minimum width=3cm, minimum height=1cm, text centered, draw=black, fill=green!30]

\title{Gov 2006: Formal Political Theory II \\
Section 10}
\date{\today}
\author{ \textbf{Sophie Hill}}


\begin{document}
  \maketitle
  
 %%%%%%%%%%%%%%%%%%%%%%%%%%%%%%%%%%%%%%%%%%
\begin{frame}{Today}

\Large

PSET practice!

\pause

\begin{itemize}
\setlength{\itemsep}{1em}
\item Political regimes: parliamentary vs presidential
\begin{itemize}
\large
\item Combining insights from prior weeks (comparative institutions + legislative bargaining)
\item Introducing core model for PSET 8 Q1 \& Q2 
\end{itemize}

\pause

\item Citizen-candidate model
\begin{itemize}
\large
\item Review the basic set-up
\item Hints for PSET 8 Q3 \& Q4 \\
\end{itemize}
\end{itemize}

\end{frame}
%%%%%%%%%%%%%%%%%%%%%%%%%%%%%%%%%%%%%%%%%%
\begin{frame}{Political regimes: PT 10.6.1}

\textbf{PT Exercise 10.6.1: A model with a prime minister}

\textbf{Set-up}

\begin{itemize}
\item 3 groups of voters $J = 1, 2, 3$, each of mass 1
\item Each group represented by a single legislator $I=1,2,3$
\item Prime minister $P$ heads the government
\end{itemize}

\pause


Voters in district $J$ have preferences

$$ w^{J}=c^{J}+H(g)=y-\tau+f^{J}+H(g) $$

where $\tau$ denotes taxes, $f^J$ denotes transfers to group $j$, and $g$ denotes a general public good benefiting all voters.
\end{frame}

%%%%%%%%%%%%%%%%%%%%%%%%%%%%%%%%%%%%%%%%%%
\begin{frame}{Political regimes: PT 10.6.1}

Rents $r_i$ to legislator $i$ are allocated via a legislative decision. Legislators care about endogenous rents as well as future exogenous rents: $v_{l}=\gamma r_{l}+p_{l} R$. \\
\vspace{1em} 

\pause

The government's budget constraint is:

$$ 3 \tau=g+\sum_{J} f^{J}+\sum_{l} r_{l}+r_{P}=g+f+r $$

where $f$ denotes aggregate transfers. 
\vspace{1em}

\pause


All items in the government budget constraint must be non-negative ($\tau, g, f, r \geq 0$). Let $\mathbf{q}$ be the full policy vector.


\end{frame}


%%%%%%%%%%%%%%%%%%%%%%%%%%%%%%%%%%%%%%%%%%
\begin{frame}{Political regimes: PT 10.6.1}

\textbf{Timing}

\begin{enumerate}
\item Voters determine their (publicly-known) reelection strategies, setting a cutoff $\varpi_J$.
\item Prime ministers proposes policy vector $\mathbf{q}$
\item Legislature votes on $\mathbf{q}$. If a majority ($n \geq 2$) support, it is implemented and the PM stays in office. If not, the PM loses office and a default policy $\mathbf{\bar{q}}$ is implemented, with $\tau = r_l = \bar{r} > 0$ and $g = f^J = 0$.
\item Voters observe the outcome of the legislative decision and all elements in the policy vector. Elections are held. 
\end{enumerate}


\end{frame}

%%%%%%%%%%%%%%%%%%%%%%%%%%%%%%%%%%%%%%%%%%
\begin{frame}{Political regimes: PT 10.6.1}

\begin{tcolorbox}
What is a ``reelection strategy'' for the voters?
\end{tcolorbox}

\pause

\textbf{Answer}: We assume that voters from the same district coordinate their strategies, but voters across districts do not cooperate.

\pause

Formally, voters set the probability of reelection $p_I$ based on a simple retrospective voting rule:

\begin{align*}
p_{l}=\left\{\begin{array}{ll}{1} & {\text { iff } \quad W^{J}(\mathbf{q}) \geq \varpi^{J} } \\ {0} & {\text { otherwise }}\end{array}\right.
\end{align*}

Voters set the cutoff $\varpi^J$ to maximize their utility.


\end{frame}
%%%%%%%%%%%%%%%%%%%%%%%%%%%%%%%%%%%%%%%%%%
\begin{frame}{Political regimes: PT 10.6.1}

\begin{tcolorbox}
What will the level of public goods provision be in this regime?
Who will end up with positive rents, and who with transfers?
\end{tcolorbox}

\pause


\textbf{Solution}:

\begin{itemize}
\item PM only needs 2 legislators
\item Given the voters' strategies ($\varpi_1, \varpi_2, \varpi_3$), the PM will pick districts with lowest cutoff, leading to Bertrand competition.
\item WLOG, assume districts 1 and 2 form the coalition. In equilibrium, we must have $\varpi_1 = \varpi_2 = \varpi^* \leq \varpi_3$.
\item Bertrand competition $\implies$ zero transfers in equilibrium.

\end{itemize}


\end{frame}

%%%%%%%%%%%%%%%%%%%%%%%%%%%%%%%%%%%%%%%%%%
\begin{frame}{Political regimes: PT 10.6.1}

\begin{tcolorbox}
What will the level of public goods provision be in this regime?
Who will end up with positive rents, and who with transfers?
\end{tcolorbox}

\textbf{Solution (cont'd)}:

\begin{itemize}
\item To get their votes, the PM needs to offer legislators 1 and 2 enough rents to make them indifferent between supporting her proposal and getting reelected, or taking the default rents and losing office: $\gamma r_{i}+R = \gamma \bar{r}$ for $i=1,2$. 
\item So $r_i^* = \operatorname{max} \{ \bar{r} - \frac{R}{\gamma}, 0 \} $ for $i = 1, 2$.
\end{itemize}

\end{frame}

%%%%%%%%%%%%%%%%%%%%%%%%%%%%%%%%%%%%%%%%%%
\begin{frame}{Political regimes: PT 10.6.1}


\textbf{Solution (cont'd)}:

The PM solves the following problem:

\begin{align*}
\operatorname{max \, \,} \gamma r_p + R & \\
s.t. \, \, \, & f_1 + y - \tau + H(g) \geq \varpi^* \\
 & f_2 + y - \tau + H(g) \geq \varpi^* \\
 & 3\tau = g + f + r
\end{align*}

\pause


$$ \implies f_1 = f_2 = 0, \, \, \, \, \, \,  \tau^* = y, \, \, \, \, \, \, H_g(g^*) \geq \frac{1}{2}, \, \, \, \, \, \, \varpi^* = H(g^*)$$



\end{frame}


%%%%%%%%%%%%%%%%%%%%%%%%%%%%%%%%%%%%%%%%%%
\begin{frame}{Political regimes: PT 10.6.1}

\begin{tcolorbox}
\textbf{Bonus question}: How does the provision of public goods in this equilibrium compare to the social optimum?
\end{tcolorbox}

\pause

\textbf{Answer}: Public goods are \alert{underprovided}. The social planner sets the marginal benefit to each group equal to the marginal social cost, i.e. $H_g(g^{opt}) = \frac{1}{3}$. \\

\vspace{1em}
Given that $H(\cdot)$ is strictly concave, we have that $$ g^* = H_g^{-1} \left( \frac{1}{2} \right) < H_g^{-1} \left( \frac{1}{3} \right) = g^{opt} $$

\end{frame}

%%%%%%%%%%%%%%%%%%%%%%%%%%%%%%%%%%%%%%%%%%
\begin{frame}{Political regimes: PT 10.6.2}

\textbf{PT Exercise 10.6.2: Adding a president}


Same set-up as before, except:

\begin{itemize}
\item we have a President, $P$, instead of a Prime Minister
\pause
\item we have two separate agenda setters: 
\begin{itemize}
\item $a_\tau$ for the ``finance committee''  
\item $a_g$ for the ``expenditure committee''
\end{itemize}
\pause
\item President can veto the allocation decision of Congress 
\end{itemize}

\end{frame}

%%%%%%%%%%%%%%%%%%%%%%%%%%%%%%%%%%%%%%%%%%
\begin{frame}{Political regimes: PT 10.6.2}

\footnotesize
\textbf{Timing}:

\begin{enumerate}
\item 2 out of the 3 legislators are appointed agenda setters for the ``finance committee'' $a_\tau$ and the ``expenditure committee'' $a_g$.
\item Voters set optimal cutoff utilities $\varpi_J$, conditional on their legislator's status.
\item $a_\tau$ proposes a tax rate, $\tau$.
\item Congress votes on the tax proposal. If it is not approved, the default tax rate is $\bar{\tau} > 0$.
\item $a_g$ proposes $g$, $f^J$ and $r_i$ (n.b. President can receive rents $r_P$ as well), subject to $3\tau \geq g + f + r$.
\item Congress votes on the allocation proposal. If it is not approved, the default allocation is $g=0, \, \, f^J \equiv \tau - r_l \geq 0, \, \, r_l = \bar{r}$.
\item President decides whether to veto the decision of the Congress. If she does, the default allocation is implemented.
\item Voters observe everything and elections are held. The president is elected in national elections, and the legislators contest in their districts. Assume $R$ to be large.
\end{enumerate}

\end{frame}

%%%%%%%%%%%%%%%%%%%%%%%%%%%%%%%%%%%%%%%%%%
\begin{frame}{Political regimes: PT 10.6.2}

\begin{tcolorbox}
\textbf{(A)} Construct an equilibrium in which public goods are provided at a level $H_g(g^*) = 1$.
\end{tcolorbox}

\pause

In order for public goods to be provided at $g^*>0$ in equilibrium, $a_g$ must want to be reelected, i.e.: 

$$ \gamma (g^*-\overline{r} ) \leq \gamma r^{*}+R $$

\pause

\alert{RHS} = equilibrium payoffs with reelection.
\newline \alert{LHS} = off-equilibrium payoffs, where $a_g$ appropriates all the taxes ($3\tau = g^*$) and buys the vote of one other legislator at cost $\bar{r}$.


\end{frame}

%%%%%%%%%%%%%%%%%%%%%%%%%%%%%%%%%%%%%%%%%%
\begin{frame}{Political regimes: PT 10.6.2}

\begin{tcolorbox}
\textbf{(A)} Construct an equilibrium in which public goods are provided at a level $H_g(g^*) = 1$.
\end{tcolorbox}

As the question specifies, we can assume $R$ is large. In particular, let's assume $R$ is large enough such that the reelection constraint holds even if equilibrium rents $r^* = 0$. \pause This implies:

$$ g^{*}-\overline{r}-\frac{R}{\gamma} \leq 0 $$


\end{frame}

%%%%%%%%%%%%%%%%%%%%%%%%%%%%%%%%%%%%%%%%%%
\begin{frame}{Political regimes: PT 10.6.2}

\begin{tcolorbox}
\textbf{(A)} Construct an equilibrium in which public goods are provided at a level $H_g(g^*) = 1$.
\end{tcolorbox}

What cutoff set by the voters would support this equilibrium?

$$ \varpi^{*}=y-\frac{g^{*}}{3}+H\left(g^{*}\right) $$

for all districts, as well as for the national presidential election.


\end{frame}

%%%%%%%%%%%%%%%%%%%%%%%%%%%%%%%%%%%%%%%%%%
\begin{frame}{Political regimes: PT 10.6.2}

\begin{tcolorbox}
\textbf{(A)} Construct an equilibrium in which public goods are provided at a level $H_g(g^*) = 1$.
\end{tcolorbox}

Backwards induction: how should $a_\tau$ set $\tau$, given how $a_g$ will allocate? 

\pause 

Given that $a_g$ will allocate all taxes to the public good, $a_\tau$ sets taxes just high enough to finance this: $\tau^* = \frac{g^*}{3}$. 

\pause

\alert{Why not set taxes higher?} \pause Because $a_g$ will appropriate anything above this in rents to herself.


\end{frame}

%%%%%%%%%%%%%%%%%%%%%%%%%%%%%%%%%%%%%%%%%%
\begin{frame}{Political regimes: PT 10.6.2}

\begin{tcolorbox}
\textbf{(B)} Show that there are an infinite number of equilibria with $H_g(g^*) = 1$ and positive transfers for the district of $a_g$.
\end{tcolorbox}

Note, in part (A), we did not explicitly specify the transfers $f_J$. In fact, there are infinite equilibria corresponding to different levels of positive transfers to the district of $a_g$:

\pause

$ \begin{aligned} 
&f^{a_{g}} =3 x, \, \, \, f^{j}=0, \text { where } x>0 \\
&\tau^{*} =\frac{g^{*}}{3}+x \\ 
&\overline{w}^{a_{g}} =y-\left(\frac{g^{*}}{3}+x\right)+H\left(g^{*}\right)+3 x=y-\frac{g^{*}}{3}+H\left(g^{*}\right)+2 x \\ 
&\overline{w}^{j} =y-\left(\frac{g^{*}}{3}+x\right)+H\left(g^{*}\right),\text{  for } j \neq a_{g}  
 \end{aligned}$

\end{frame}

%%%%%%%%%%%%%%%%%%%%%%%%%%%%%%%%%%%%%%%%%%
\begin{frame}{Political regimes: PT 10.6.2}

\begin{tcolorbox}
\textbf{(B)} Show that there are an infinite number of equilibria with $H_g(g^*) = 1$ and positive transfers for the district of $a_g$.
\end{tcolorbox}

\alert{Why is $a_g$ able to extract these transfers? }

\pause


\begin{itemize}
\item Because $a_g$ is always in the coalition that supports the policy in equilibrium.
\item So the other two legislators end up in Bertrand competition.
\item They have to go along with ${a_{g}}^\prime s$ proposal if they want any public goods $g$.
\end{itemize}

 \end{frame}
 
 %%%%%%%%%%%%%%%%%%%%%%%%%%%%%%%%%%%%%%%%%%
\begin{frame}{Political regimes: PT 10.6.2}

\begin{tcolorbox}
\textbf{(C)} Compare the results of the model with a president and the model without one. Why does the addition of the president not change the equilibria?
\end{tcolorbox}


Recall that we also did not explicitly consider the presidential veto in our derivation of the equilibrium above!  \pause 

\alert{Why doesn't the veto matter?} \pause 

\begin{itemize}
\item Because the legislative process already requires the support of voters in two out of the three districts (i.e., a national majority). 
\item $P$ prefers not to veto since she will be guaranteed reelection (and thus future exogenous rents $R$)
\end{itemize} 

\end{frame}
 
 
 \begin{frame}{PSET 8}
 
 Looking ahead to \alert{PSET 8}...
 
 \begin{itemize}
 \item Q1 considers an infinitely-repreated version of the presidential model 
 \item Q2 looks at a particular subgame of the parliamentary model, when the governing coalition breaks down
 \end{itemize}
 
 \end{frame}
 
 \begin{frame}{Review: citizen-candidate models}
 
 Let's review a simplified version of a citizen-candidate model...
 
 \pause

 \textbf{Set-up}
 
 \begin{itemize}
 \item Citizens are differentiated by income $y_i$
 \item Any citizen can run as a candidate at cost $\varepsilon$
 \item Incumbents set the level of taxes (equivalently, public goods), subject to the budget constraint: $\tau y = g$
 \end{itemize}
 
 \end{frame}
 
 \begin{frame}{Review: citizen-candidate models}
 
 Let's review a simplified version of a citizen-candidate model...
 
 \textbf{Timing}:
 
 \begin{enumerate}
 \item Citizens choose whether to run.
 \item An election is held. Each citizen votes to maximize their expected utility, given how everyone else votes. Candidate with a plurality wins, ties resolved with a coin toss.
 \item Elected candidate sets policy $g^P$. If nobody runs, a default policy $g$ is implemented.
 \end{enumerate}
 
 
 \end{frame}
 
 \begin{frame}{Review: citizen-candidate models}
 
 Solve with... \pause backwards induction! ($^{\circ} \msquare ^{\circ}$)
 
 \pause
 
What policy does the winning candidate choose? \pause No commitment, so pick your bliss point:

\vspace{-1em}

\begin{align*}
u_i &= y^i (1 - \tau) + H(g) \\
&= y^i \left( 1 - \frac{g}{y} \right) + H(g) \\
\operatorname{FOC:}  & -\frac{y^i}{y} + H_g(g) = 0 \\
\implies g^P &= H_g^{-1} \left( \frac{y^P}{y} \right)
\end{align*}
 
 \end{frame}
 
  \begin{frame}{Review: citizen-candidate models}
  
  Next: how do citizens vote? \\
  
  \pause

  
  \begin{itemize}
  \item Citizens can arrange candidates according to the distance between their bliss points
  \item So in a 1- or 2-candidate election, the candidate who wins the median voter wins the election
  \item In a $3+$ candidate election, the median voter may not be pivotal since we have assumed that citizens vote strategically
  \end{itemize}
 
 
 \end{frame}
 
  \begin{frame}{Review: citizen-candidate models}

Finally: who chooses to run?

\pause


\begin{itemize}
\item A citizen only chooses to run if running gives a higher expected utility, net of entry costs, than not running, given other citizens' entry decisions
\end{itemize}

\end{frame}

 \begin{frame}{PSET 8}
 
 Looking ahead to \alert{PSET 8}...
 
 \begin{itemize}
 \item \textcolor{gray}{Q1 considers an infinitely-repreated version of the presidential model }
 \item \textcolor{gray}{Q2 looks at a particular subgame of the parliamentary model, when the governing coalition breaks down }
 \item Q3 looks at equilibria of the citizen-candidate model for specific values of $\varepsilon$
 \item Q4 explores what happens in the citizen-candidate model when voters' preferences do not satisfy the single-crossing property
 \end{itemize}
 
 \end{frame}



\end{document}
